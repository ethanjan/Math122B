\documentclass[12pt]{article}
 
\usepackage[margin=1in]{geometry}
\usepackage{amsmath,amsthm,amssymb}
\usepackage{mathtools}
\DeclarePairedDelimiter{\ceil}{\lceil}{\rceil}
%\usepackage{mathptmx}
\usepackage{accents}
\usepackage{comment}
\usepackage{graphicx}
\usepackage{IEEEtrantools}
 \usepackage{float}
 
\newcommand{\N}{\mathbb{N}}
\newcommand{\Z}{\mathbb{Z}}
\newcommand{\R}{\mathbb{R}}
\newcommand{\Q}{\mathbb{Q}}
\newcommand*\conj[1]{\bar{#1}}
\newcommand*\mean[1]{\bar{#1}}
\newcommand\widebar[1]{\mathop{\overline{#1}}}


\newcommand{\cc}{{\mathbb C}}
\newcommand{\rr}{{\mathbb R}}
\newcommand{\qq}{{\mathbb Q}}
\newcommand{\nn}{\mathbb N}
\newcommand{\zz}{\mathbb Z}
\newcommand{\aaa}{{\mathcal A}}
\newcommand{\bbb}{{\mathcal B}}
\newcommand{\rrr}{{\mathcal R}}
\newcommand{\fff}{{\mathcal F}}
\newcommand{\ppp}{{\mathcal P}}
\newcommand{\eps}{\varepsilon}
\newcommand{\vv}{{\mathbf v}}
\newcommand{\ww}{{\mathbf w}}
\newcommand{\xx}{{\mathbf x}}
\newcommand{\ds}{\displaystyle}
\newcommand{\Om}{\Omega}
\newcommand{\dd}{\mathop{}\,\mathrm{d}}
\newcommand{\ud}{\, \mathrm{d}}
\newcommand{\seq}[1]{\left\{#1\right\}_{n=1}^\infty}
\newcommand{\isp}[1]{\quad\text{#1}\quad}
\newcommand*\diff{\mathop{}\!\mathrm{d}}

\DeclareMathOperator{\imag}{Im}
\DeclareMathOperator{\re}{Re}
\DeclareMathOperator{\diam}{diam}
\DeclareMathOperator{\Tr}{Tr}
\DeclareMathOperator{\cis}{cis}

\def\upint{\mathchoice%
    {\mkern13mu\overline{\vphantom{\intop}\mkern7mu}\mkern-20mu}%
    {\mkern7mu\overline{\vphantom{\intop}\mkern7mu}\mkern-14mu}%
    {\mkern7mu\overline{\vphantom{\intop}\mkern7mu}\mkern-14mu}%
    {\mkern7mu\overline{\vphantom{\intop}\mkern7mu}\mkern-14mu}%
  \int}
\def\lowint{\mkern3mu\underline{\vphantom{\intop}\mkern7mu}\mkern-10mu\int}




\newenvironment{theorem}[2][Theorem]{\begin{trivlist}
\item[\hskip \labelsep {\bfseries #1}\hskip \labelsep {\bfseries #2.}]}{\end{trivlist}}
\newenvironment{lemma}[2][Lemma]{\begin{trivlist}
\item[\hskip \labelsep {\bfseries #1}\hskip \labelsep {\bfseries #2.}]}{\end{trivlist}}
\newenvironment{exercise}[2][Exercise]{\begin{trivlist}
\item[\hskip \labelsep {\bfseries #1}\hskip \labelsep {\bfseries #2.}]}{\end{trivlist}}
\newenvironment{problem}[2][Problem]{\begin{trivlist}
\item[\hskip \labelsep {\bfseries #1}\hskip \labelsep {\bfseries #2.}]}{\end{trivlist}}
\newenvironment{question}[2][Question]{\begin{trivlist}
\item[\hskip \labelsep {\bfseries #1}\hskip \labelsep {\bfseries #2.}]}{\end{trivlist}}
\newenvironment{corollary}[2][Corollary]{\begin{trivlist}
\item[\hskip \labelsep {\bfseries #1}\hskip \labelsep {\bfseries #2.}]}{\end{trivlist}}

\newenvironment{solution}{\begin{proof}[Solution]}{\end{proof}}
 
\begin{document}
 
% --------------------------------------------------------------
%                         Start here
% --------------------------------------------------------------
\title{Math 122B Homework 4}
\author{Ethan Martirosyan}
\date{\today}
\maketitle
\hbadness=99999
\hfuzz=50pt
\section*{Problem 6}
First, we let the mapping $f$ be defined as follows:
\[
f(z) = \frac{-(z-1)^2}{4(z+1)^2}
\] From the textbook, we know that this mapping takes the upper semi-disk to the upper half plane. Next, we let the mapping $g$ be defined by
\[
g(z) = \frac{i-z}{i+z}
\] This mapping takes the upper half plane to the unit disk. Thus the desired mapping is $g \circ f$.
\newpage
\section*{Problem 12}
Let us consider the mapping $g = f_1 \circ f_2^{-1}$. This function $g$ is an automorphism of the unit disk. Notice that
\[
g(0) = f_1(f_2^{-1}(0)) = f_1(z_0) = 0
\] so that $g(z) = e^{i\theta} z$. Next, we have
\[
e^{i\theta} = g^\prime(0) = f_1^\prime(f_2^{-1}(0)) ((f_2^{-1})^\prime(0)) = f_1^\prime(z_0) \cdot \frac{1}{f_2^\prime(f_2^{-1}(0))} = \frac{f_1^\prime(z_0)}{f_2^\prime(z_0)} > 0 
\] so that $e^{i\theta} = 1$ and $g$ is the identity. From this, we deduce that $f_1 = f_2$. 
\newpage
\section*{Problem 13}
First, we suppose that $f$ is a conformal mapping from some half-plane $H_1$ to another half-plane $H_2$. Let $g$ be a linear mapping that takes $H_1$ to the upper half-plane, and let $h$ be a linear mapping that takes $H_2$ to the upper half-plane. Then $h \circ f \circ g^{-1}$ is an automorphism of the upper half-plane, so we may write
\[
h \circ f \circ g^{-1} = \frac{az+b}{cz+d}
\] where $a,b,c,d$ are real and $ad - bc > 0$. Then, we obtain
\[
f = h^{-1} \circ \frac{az+b}{cz+d} \circ g
\] so that $f$ is bilinear (since the composition of bilinear transformations is bilinear).

Next, we may suppose that $f$ is a conformal mapping of the half-plane $H$ to the disk $D$. Let $g$ be a linear mapping from $H$ to the upper half-plane, and let $h$ be a linear mapping from the unit disk to $D$. Then $h^{-1} \circ f \circ g^{-1}$ is a conformal mapping of the upper half-plane onto $U$ so that
\[
h^{-1} \circ f \circ g^{-1} = e^{i\theta} \bigg(\frac{z-\alpha}{z-\overline{\alpha}}\bigg)
\] for some $\alpha$ with $\imag \alpha > 0$. Thus we obtain
\[
f = h \circ e^{i\theta} \bigg(\frac{z-\alpha}{z-\overline{\alpha}}\bigg) \circ g
\] so that $f$ is bilinear.

Next, we may suppose that $f$ is a conformal mapping of the disk $D$ to the half-plane $H$. Let $g$ be a linear mapping of the half-plane $H$ to the upper half-plane. Let $h$ be a mapping of the unit disk to $D$. Then $h^{-1} \circ f^{-1} \circ g^{-1}$ maps the upper half-plane to the unit disk, so it is of the form
\[
h^{-1} \circ f^{-1} \circ g^{-1} = e^{i\theta} \bigg(\frac{z-\alpha}{z-\overline{\alpha}}\bigg)
\] so that
\[
f^{-1} = h \circ e^{i\theta} \bigg(\frac{z-\alpha}{z-\overline{\alpha}}\bigg) \circ g
\] Thus $f^{-1}$ is bilinear so that $f$ is bilinear.

Finally, we may suppose that $f$ is a conformal mapping of the disk $D_1$ onto the disk $D_2$. Let $g$ be a linear mapping from the unit disk to $D_1$, and let $h$ be a linear mapping from $D_2$ to the unit disk. Then $h \circ f \circ g$ is an automorphism of the unit disk. Thus we have
\[
h \circ f \circ g = e^{i\theta}z
\] so that
\[
f = h^{-1} \circ e^{i\theta} z \circ g^{-1}
\] is a bilinear map.
\newpage
\section*{Problem 14}
First, we note that
\[
f(z) = \frac{az+b}{cz+d} = \frac{az+b}{cz+d} \cdot \frac{c \overline{z} + d}{c \overline{z} + d} = \frac{ac \vert z \vert^2 + bc \overline{z} + adz + bd}{\vert cz + d \vert^2}
\] From this, we obtain
\[
\imag f(z) = (ad - bc) \imag z
\] If $\imag z > 0$, then we know that
\[
\imag f(z) = (ad - bc) \imag z < 0
\] since $ad - bc < 0$. Thus we find that $f$ maps the upper half-plane into the lower half-plane. Next, we claim that $f$ is onto. Let $z \in \cc$ be such that $\imag z < 0$. Then, we have
\[
f^{-1}(z) = \frac{dz-b}{-cz+a} = \frac{dz-b}{-cz+a} \cdot \frac{-c\overline{z}+a}{-c\overline{z}+a} = \frac{-cd\vert z\vert^2 - ab + bc\overline{z} + adz}{\vert - cz + a \vert^2}
\] We have
\[
\imag f^{-1}(z) = (ad - bc) \imag z > 0 
\] since $ad - bc < 0$ and $\imag z < 0$. Thus we find that $f$ maps the upper half-plane onto the lower half-plane.
\newpage
\section*{Problem 15}
Let $g$ be an automorphism of the first quadrant. By Lemma $13.13$, we may write $g = f^{-1} \circ h \circ f$, where $f$ is a conformal mapping of the first quadrant onto the upper half-plane and $h$ is an automorphism of the upper half-plane. Let $f(z) = z^2$. By Theorem $13.17$, we know that
\[
h(z) = \frac{az+b}{cz+d}
\] where $a,b,c,d$ are real and $ad-bc > 0$. Thus, we find that
\[
g(z) = \sqrt{\frac{az^2+b}{cz^2+d}}
\]
\newpage
\section*{Problem 18}
First, we suppose that $z_4$ lies on the same circle or line as $z_1$, $z_2$, and $z_3$. Then, we claim that $[z_1,z_2,z_3,z_4]$ is real-valued. Let $T$ be the bilinear mapping that sends $z_1$, $z_2$ and $z_3$ to $\infty$, $0$, and $1$. Notice that $T$ sends the circle or line containing $z_1$, $z_2$, and $z_3$ to the real line (since bilinear maps take circles and lines to circles and lines). Thus $z_4$ is also mapped to a point on the real line so that $T(z_4) = [z_1,z_2,z_3,z_4]$ is real-valued. 

Conversely, we suppose that  $[z_1,z_2,z_3,z_4]$ is real-valued, and we claim that $z_4$ must lie on the same circle or line as $z_1$, $z_2$, and $z_3$. Let $T$ be the bilinear mapping that sends $z_1$, $z_2$, and $z_3$ to $\infty$, $0$, and $1$. By assumption, we have $T(z_4) = [z_1,z_2,z_3,z_4] \in \rr$. Thus, we know that $T(z_4)$ is on the real line along with $T(z_1)$, $T(z_2)$, and $T(z_3)$. Applying $T^{-1}$, which is bilinear, we find that $z_4$ is on the same circle or line as $z_1$, $z_2$, and $z_3$.
\end{document} 